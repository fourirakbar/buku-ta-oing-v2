\chapter{INSTALASI PERANGKAT LUNAK}

\section*{Instalasi Pustaka Python} \label{install:pythonlibrary}
	Dalam pengembangan sistem ini, digunakan berbagai pustaka pendukung. Pustaka pendukung yang digunakan merupakan pustaka untuk bahasa pemrograman Python. Berikut adalah daftar pustaka yang digunakan dan cara pemasangannya:
    \begin{itemize}
	 \item Python Pip\\
		 \$ \texttt{sudo apt-get install python-pip}
    \item Python Dev \\
    	\$ \texttt{sudo apt-get install python-dev}
    \item Setuptools \\
    	\$ \texttt{sudo apt-get install python-setuptools}
    \item MySQLd \\
    	\$ \texttt{sudo apt-get install python-mysqldb}
    \end{itemize}
	
\section*{Pemasangan kerangka kerja Flask}
Dalam pengembangan sistem ini, digunakan Flask karena Flask merupakan kerangka kerja yang menggunakan bahasa pemrograman Python. Untuk memasang Flask, jalankan perintah \texttt{sudo pip3 install flask}.

\section*{Pemasangan perangkat lunak Gunicorn}
Dalam pengembangan sistem ini, digunakan Gunicorn untuk menangani servis dari halaman \textit{login}. Untuk memasang Gunicorn, jalankan perintah \texttt{sudo pip3 install gunicorn}.

\section*{Pemasangan perangkat lunak Supervisor}
Dalam pengembangan sistem ini, digunakan Supervisor supaya servis pada halaman \textit{login} dapat langsung berjalan ketika \textit{server} dinyalakan. Untuk memasang Supervisor, jalankan perintah \texttt{sudo apt-get install supervisor}.

\section*{Pemasangan perangkat lunak Nginx}
Dalam pengembangan sistem ini, digunakan Nginx sebagai web \textit{server} untuk halaman \textit{login}. Untuk memasang Nginx, jalankan perintah \texttt{sudo apt-get install nginx}.

\section*{Instalasi Lingkungan Docker}
Proses pemasangan Docker dpat dilakukan sesuai tahap berikut:
\begin{itemize}
	\item Menambahkan repository Docker\\
	Langkah ini dilakukan untuk menambahkan \textit{repository} Docker ke dalam paket \texttt{apt} agar dapat di unduh oleh Ubuntu. Untuk melakukannya, jalankan perintah berikut:
	\begin{tabbing}
		\texttt{sudo apt-get -y install \char`\\} \\
		\hspace{5 mm} \texttt{apt-transport-https \char`\\} \\
		\hspace{5 mm} \texttt{ca-certificates \char`\\} \\
		\hspace{5 mm} \texttt{curl} \\
		\\
		\texttt{curl -fsSL https://download.docker.com/linux/} \\
		\hspace{7 mm} \texttt{ubuntu/gpg | sudo apt-key add -} \\
		\\
		\texttt{sudo add-apt-repository \char`\\} \\
		\hspace{7 mm} \texttt{"deb [arch=amd64] https://download.docker.com/} \\
		\hspace{9 mm} \texttt{linux/ubuntu \char`\\} \\
		\hspace{7 mm} \texttt{\$ (lsb\_release -cs) \char`\\} \\
		\hspace{7 mm} \texttt{stable"} \\
		\\
		\texttt{sudo apt-get update} \\
	\end{tabbing}
	
	\item Mengunduh Docker \\
	Docker dikembangkan dalam dua versi, yaitu CE (\textit{Community Edition}) dan EE (\textit{Enterprise Edition}). Dalam pengembangan sistem ini, digunakan Docker CE karena merupakan versi Docker yang gratis. Untuk mengunduh Docker CE, jalankan perintah \texttt{sudo apt-get -y install docker-ce}.
	
	\item Mencoba menjalankan Docker \\
	Untuk melakukan tes apakah Docker sudah terpasang dengan benar, gunakan perintah \texttt{sudo docker run hello-world}.
\end{itemize}