\chapter{PENDAHULUAN}
Pada bab ini akan dipaparkan mengenai garis besar Tugas Akhir yang meliputi latar belakang, tujuan, rumusan dan batasan permasalahan, metodologi pembuatan Tugas Akhir, dan sistematika penulisan.

\section{Latar Belakang}
Seiring dengan perkembangan zaman yang sangat pesat, negara-negara sudah mempunyai teknologi yang sangat maju. Teknologi mempunyai peranan yang sangat penting dalam kehidupan manusia, karena dengan adanya teknologi, manusia bisa saling berhubungan dengan mudah. Sekarang teknologi sudah semakin canggih. Teknologi yang paling populer sekarang ini adalah internet karena dengan adanya internet, banyak informasi-informasi yang dapat kita ambil dengan mudah. Internet merupakan suatu perpustakaan besar yang di dalamnya terdapat sangat banyak informasi yang berupa teks dalam bentuk media elektronik. Selain itu internet dikenal sebagai dunia maya, karena hampir seluruh aspek kehidupan di dunia nyata ada di internet, seperti olah raga, politik, hiburan, akademik, bisnis, dan lain sebagainya. Internet juga mempunyai peranan yang sangat penting dalam dunia pendidikan, karena dengan adanya internet bisa menambah ilmu pengetahuan kita dan dapat menambah motivasi belajar siswa ataupun mahasiswa. Dengan dimanfaatkan internet dalam dunia pendidikan agar siswa atau mahasiswa dapat memiliki komitmen untuk belajar secara aktif dan memiliki teknis kemampuan khususnya di bidang pendidikan. Oleh karena itu, internet dapat mempermudah proses belajar mengajar dengan baik.

Dalam dunia pendidikan, internet telah menjadi \textit{platform} penting, misalnya adalah proses belajar mengajar yang dilakukan secara \textit{online} dengan e-learning, ataupun ketika pengajar memberikan nilai kepada siswanya dilakukan secara \textit{online}, dan lain sebagainya. Keamanan menjaga data-data dalam dunia pendidikan, melindungi terhadap penggunaan malware, menerapkan kepatuhan akses internet, menyederhanakan manajemen jaringan menjadi tantangan utama untuk manajemen TI. Maka dari itu dibutuhkan sebuah \textit{online} yang digunakan sebagai \textit{internet access management} atau untuk melakukan manajemen akses terhadap \textit{user} yang menggunakan jaringannya.

Namun akan terjadi permasalahan ketika banyak \textit{user} yang mengakses internet dengan menggunakan \textit{server} yang digunakan sebagai \textit{internet access management}. Dalam pembacaan \textit{log history} dari setiap \textit{user} akan tercampur karena hanya melewati satu \textit{server} saja. 

Dalam tugas akhir ini akan dibuat sebuah rancangan sistem pada \textit{server} yang akan dijadikan sebagai \textit{internet access management}, yang memungkinkan untuk mencatat setiap \textit{log history} dari setiap \textit{user} yang mengakses internet secara detail. Rancangan sistem pada \textit{server} akan menggunakan kontainer \textit{docker}. Kontainer \textit{docker} merupakan \textit{operating-system-level virtualization} untuk menjalankan beberapa sistem linux yang terisolasi (kontainer) pada sebuah host. Kontainer berfungsi untuk mengisolasi aplikasi atau servis dan dependensinya. Untuk setiap servis atau aplikasi yang terisolasi dibutuhkan satu kontainer pada \textit{server host} yang ada dan setiap kontainer akan menggunakan sumber daya yang ada pada \textit{server host} selama kontainer tersebut menyala.

\section{Rumusan Masalah}
Berikut beberapa hal yang menjadi rumusan masalah dalam tugas akhir ini:
\begin{enumerate}
	\item Bagaimana cara \textit{client} untuk melakukan \textit{autentifikasi}?
	\item Bagaimana cara membuat sebuah kontainer \textit{docker} secara otomatis ketika terdapat \textit{client} yang akan mengakses internet?
	\item Bagaimana cara mengarahkan \textit{traffic} dari \textit{client} ke kontainer \textit{docker} yang sesuai?
	\item Bagaimana cara mencatat aktivitas dari \textit{client}?
	\item Bagaimana perbandingan performa antara IAM konvensional dengan IAM berbasi kontainer?
	\item Bagaimana mengevaluasi penggunaan sumber daya dan skalabilitas pada \textit{docker host}?
\end{enumerate}

\section{Batasan Masalah}
Dari permasalahan yang telah diuraikan di atas, terdapat beberapa batasan masalah pada tugas akhir ini, yaitu:
\begin{enumerate}
	\item Satu \textit{client} yang berhasil \textit{login} akan disediakan satu kontainer \textit{docker}.
	\item Kontainer yang digunakan adalah \textit{docker}.
	\item Parameter untuk mengetahui apa saja yang diakses oleh \textit{client} adalah \textit{access log} dari \textit{client} tersebut.
	\item Setiap \textit{client} mendapatkan IP \textit{private}.
	\item Performa yang diukur adalah \textit{response time}.
	\item Bahasa pemrograman yang digunakan adalah \textit{Python}.
\end{enumerate}

\section{Tujuan}
Tugas akhir dibuat dengan beberapa tujuan. Berikut beberapa tujuan dari pembuatan tugas akhir:
\begin{enumerate}
	\item Mengetahui cara bagaimana \textit{client} dapat melakukan \textit{autentifikasi}.
	\item Mengimplementasikan metode untuk membuat sebuah kontainer terhadap \textit{client} yang telah berhasil \textit{login} ke jaringan ITS.
	\item Mengetahui cara untuk mengarahkan \textit{traffic} dari \textit{client} ke kontainer \textit{docker} yang sesuai.
	\item Mengetahui bagaimana cara mencatat aktivitas \textit{client}.
	\item Mengetahui penggunaan sumber daya dan skalabilitas pada \textit{docker host}.
\end{enumerate}

\section{Manfaat}
Tugas akhir dibuat dengan beberapa manfaat. Berikut beberapa manfaat dari pembuatan tugas akhir:
\begin{enumerate}
	\item Mengetahui cara bagaimana \textit{client} dapat melakukan \textit{autentifikasi}.
	\item Mengethaui cara untuk mengarahkan \textit{traffic} dari \textit{client} ke kontainer \textit{docker} yang sesuai.
	\item Mempermudah pencatatan aktivitas dari masing-masing \textit{client} yang mengakses internet.
	\item Meringankan beban dari penggunaan \textit{server} di ITS karena penggunaan kontainer \textit{docker} lebih ringan.
\end{enumerate}      

\section{Metodologi}
Metodologi yang digunakan pada pengerjaan Tugas Akhir ini
adalah sebagai berikut:
\subsection{Studi literatur}
Studi literatur merupakan langkah yang dilakukan untuk mendukung dan memastikan setiap tahap pengerjaan tugas akhir sesuai dengan standar dan konsep yang berlaku. Pada tahap studi literatur ini, akan dilakukan studi mendalam mengenai kontainer \textit{docker}, \textit{flask}, \textit{mitmproxy}, dan pembuatan aturan dengan menggunakan \textit{iptables}. Adapun literatur yang dijadikan sumber berasal dari paper, buku, materi perkuliahan, forum serta artikel dari internet.

\subsection{Desain dan Perancangan Sistem}
Tahap ini meliputi perancangan sistem berdasarkan studi literatur dan pembelajaran konsep. Tahap ini merupakan tahap yang paling penting dimana bentuk awal aplikasi yang akan diimplementasikan didefinisikan. Pada tahapan ini dibuat kasus penggunaan yang ada pada sistem, arsitektur sistem, serta perencanaan implementasi pada sistem.
\subsection{Implementasi Sistem}
Implementasi merupakan tahap membangun implementasi rancangan sistem yang telah dibuat. Pada tahapan ini merealisasikan apa yang telah didesain dan dirancang pada tahapan sebelumnya, sehingga menjadi sebuah sistem yang sesuai dengan apa yang telah direncanakan.
\subsection{Uji Coba dan Evaluasi}
Pada tahapan ini dilakukan uji coba terhadap sistem yang telah dibuat. Pengujian dan evaluasi akan dilakukan dengan melihat kesesuaian dengan perencanaan. Selain itu, tahap ini juga akan melakukan uji performa sistem dan melakukan perbandingan dengan metode lain untuk mengetahui efisiensi penggunaan sumber daya serta evaluasi berdasarkan hasil uji performa tersebut. 

\section{Sistematika Laporan}
Buku tugas akhir ini bertujuan untuk mendapatkan gambaran dari pengerjaan tugas akhir ini. Selain itu, diharapkan dapat berguna bagi pembaca yang berminat melakukan pengambangan lebih lanjut. Secara garis besar, buku tugas akhir ini terdiri atas beberapa bagian seperti berikut:
\begin{enumerate}
	
	\item \textbf{Bab I} \indent \textbf{Pendahuluan} \\        
	\indent \indent Bab yang berisi latar belakang, tujuan, manfaat, permasalahan, batasan masalah, metodologi yang digunakan dan sistematika laporan.
	\\
	\item \textbf{Bab II} \indent \textbf{Dasar Teori}
	\\
	\indent \indent Bab ini berisi penjelasan secara detail mengenai dasar-dasar penunjang dan teori-teori yang yang digunakan dalam pembuatan tugas akhir ini.
	\\
	\item \textbf{Bab III} \indent \textbf{Desain dan Perancangan}
	\\
	\indent \indent Bab ini berisi tentang analisis dan perancangan sistem yang dibuat, termasuk di dalamnya mengenai analisis kasus penggunaan, desain arsitektur sistem, dan perancangan implementasi sistem.
	\\
	\item \textbf{Bab IV} \indent \textbf{Implementasi}
	\\
	\indent \indent Bab ini membahas implementasi dari desain yang telah dibuat pada bab sebelumnya. Penjelasan berupa pemasangan alat dan kode program yang digunakan untuk mengimplementasikan sistem.
	\\
	\item \textbf{Bab V} \indent \textbf{Uji Coba dan Evaluasi}
	\\
	\indent \indent Bab ini membahas tahap-tahap uji coba serta melakukan evaluasi terhadap sistem yang dibuat.
	\\
	\item \textbf{Bab VI} \indent \textbf{Kesimpulan dan Saran}
	\\
	\indent \indent Bab ini merupakan bab terakhir yang memberikan kesimpulan dari hasil percobaan dan evaluasi yang telah dilakukan. Pada bab ini juga terdapat saran bagi pembaca yang berminat untuk melakukan pengembangan lebih lanjut.    
\end{enumerate}