\chapter{Konfigurasi}

\section*{Konfigurasi Supervisor}

Supaya servis pada halaman \textit{login} dapat langsung berjalan, penulis menambahkan konfigurasi perangkat lunak Supervisor pada \texttt{/etc/supervisor/conf.d/app.conf} sepert pada Kode Sumber \ref{appconf}. 

\begin{lstlisting}[frame=single,tabsize=2,breaklines,caption={Isi Berkas app.conf},label=appconf, captionpos=b]
command = /home/fourirakbar/flask-loginpage/flask-loginpageenv/bin/python /home/fourirakbar/flask-loginpage/flask-loginpageenv/bin/gunicorn -b 0.0.0.0:4000 app:app

directory = /home/fourirakbar/flask-loginpage

user = fourirakbar

stdout_logfile = /home/fourirakbar/flask-loginpage/logs/app_stdout.log

stderr_logfile = /home/fourirakbar/flask-loginpage/logs/app_stderr.log

redirect_stderr = True
autostart = True
enviroment = PATH = "/home/fourirakbar/flask-loginpage/flask-loginpageenv/bin", 

PRODUCTION=1
\end{lstlisting}

\section*{Konfigurasi Nginx}

Supaya web \textit{server} untuk halaman \textit{login} dapat dijalankan, penulis menambahkan konfigurasi perangkat lunak Nginx pada \texttt{/etc/nginx/sites-available/app} seperti pada Kode Sumber \ref{nginxapp}.

\begin{lstlisting}[frame=single,tabsize=2,breaklines,caption={Isi Berkas app},label=nginxapp, captionpos=b]
server {
	listen 80;
	server_name 10.151.36.173;
	
	add_header X-Frame-Options SAMEORIGIN;
	add_header X-Content-Type-Options nosniff;
	add_header X-XSS-Protection "1; mode=block";
	
	access_log /home/fourirakbar/flask-loginpage/logs/app_access.log;
	location / {
		proxy_pass       http://127.0.0.1:4000;
		proxy_set_header Host        $host;
		proxy_set_header X-Real-IP   $remote_addr;
		proxy_set_header X-Forwarded-For $proxy_add_x_forwarded_for;
	}
	location ^~ /static/  {
	include     /etc/nginx/mime.types;
	alias       /home/fourirakbar/flask-loginpage/static/;
	}  
}
\end{lstlisting}