\begin{abstrak}
Internet turut memiliki peranan penting dalam dunia pendidikan, karena dengan adanya internet dapat menambah ilmu pengetahuan dan menambah motivasi belajar siswa ataupun mahasiswa. Dalam dunia pendidikan, internet telah menjadi platform penting, misalnya adalah proses belajar mengajar yang dilakukan secara online dengan e-learning, ataupun ketika pengajar memberikan nilai kepada siswanya dilakukan secara online, dan lain sebagainya. Keamanan menjaga data-data dalam dunia pendidikan, melindungi terhadap penggunaan malware, menerapkan kepatuhan akses internet, menyederhanakan manajemen jaringan menjadi tantangan utama untuk manajemen TI. Maka dari itu dibutuhkan sebuah online yang digunakan sebagai internet access management atau untuk melakukan manajemen akses terhadap user yang menggunakan jaringannya. Namun akan terjadi permasalahan ketika banyak user yang mengakses internet dengan menggunakan server yang digunakan sebagai internet access management. Dalam pembacaan log history dari setiap user akan tercampur karena hanya melewati satu server saja.

Internet Access Management Berbasis Kontainer  memungkinkan untuk mencatat setiap log history dari setiap user yang mengakses internet secara detail. Rancangan sistem pada server akan menggunakan kontainer docker. Kontainer docker merupakan operating-system-level virtualization untuk menjalankan beberapa sistem linux yang terisolasi (kontainer) pada sebuah host. Kontainer berfungsi untuk mengisolasi aplikasi atau servis dan dependensinya. Untuk setiap servis atau aplikasi yang terisolasi dibutuhkan satu kontainer pada server host yang ada dan setiap kontainer akan menggunakan sumber daya yang ada pada server host selama kontainer tersebut menyala. Pembuatan Internet Access Management Berbasis Kontainer dapat mengetahui cara bagaimana client dapat melakukan autentifikasi, untuk mengarahkan traffic dari client ke kontainer docker yang sesuai. Dapat juga digunakan untuk mempermudah pencatatan aktivitas dari masing-masing client yang mengakses internet dan juga dapat meringankan beban dari penggunaan server.\\

	\noindent \textbf{Kata-Kunci}: Docker, Internet Access Management, Kontainer
\end{abstrak}
\newpage
\begin{abstract}
Internet has an important role in the education, because we can explore knowledge and increase student motivation or student learning. In education, the Internet has become an important platform, for example is the process of teaching and learning that is done online with e-learning, or when teachers give score to their students are done online, and so on. The security of keeping the data in education, protecting against malware, using internet access compliance, simplifying network management becomes a major challenge for IT management. Therefore we need an online system that is used as internet access management or to perform management access to users who use the network. But there will be problems when many users access the internet with a server that is used as internet access management. The process of reading log history of each user will be mixed because it only passes one server only.

Container Based Internet Access Management makes it possible to log every log history of every user accessing the Internet in detail. The system design on the server will use docker containers. Docker containers are operating-system-level virtualization to run some isolated linux systems (containers) on a host. Containers serve to isolate applications or services and their dependencies. For each isolated service or application it takes one container on an existing host server and each container will use the resources on the host server as long as the container is on. Creation of Container Based Internet Access Management can find out how the client can authenticate, to redirect traffic from the client to the appropriate docker container. It can also be used to facilitate the recording of activity from each client accessing the internet and can also lighten the the server load usage.\\

	\noindent \textbf{Keywords}: Container, Docker, Internet Access Management
\end{abstract}