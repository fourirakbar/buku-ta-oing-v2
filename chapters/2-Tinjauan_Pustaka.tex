\chapter{TINJAUAN PUSTAKA}
\section{Python}
Python adalah bahasa pemrograman interpretatif multiguna dengan prinsip agar sumber kode yang dihasilkan memiliki tingkat keterbacaan yang baik. Python diklaim sebagai bahasa yang menggabungkan kapabilitas, kemampuan, dengan sintaksis kode yang sangat jelas, dan dilengkapi dengan fungsionalitas pustaka standar yang besar serta komprehensif. Python mendukung beragam paradigma pemrograman, seperti pemrograman berorientasi objek, pemrograman imperatif, dan pemrograman fungsional. Python dapat digunakan untuk berbagai keperluan pengembangan perangkat lunak dan dapat berjalan di berbagai platform sistem operasi. \cite{bab2-python}

\section{Flask}
Flask adalah sebuah kerangka kerja web. Artinya, Flask menyediakan perangkat, pustaka, dan teknologi yang memungkinkan seorang pengembang untuk membangun aplikasi berbasis web. Aplikasi web yang bisa dibangun bisa berupa sebuah halaman web, blog, wiki, bahkan untuk web komersial. Flask dibangun berbasiskan pada Werkzeug, Jinja 2, dan MarkupSafe yang mana menggunakan bahasa pemrograman Python sebagai basisnya. Flask sendiri pertama kali dikembangkan pada tahun 2010 dan didistribusikan dengan lisensi BSD. \cite{bab2-flask} \\
\indent Flask termasuk sebagai perangkat kerja mikro karena tidak membutuhkan banyak perangkat atau pustaka tertentu agar bisa bekerja. Flask tidak menyediakan fungsi untuk melakukan interaksi dengan basis data, tidak mempunya validasi \textit{form} atau fungsi lain yang umumnya bisa digunakan dan disediakan pada sebuah kerangka kerja web. Meskipun memiliki kemampuan yang minim, tapi Flask mendukung dan memberikan kemudahan bagi pengembang untuk menambahkan pustaka sendiri untuk mendukung aplikasinya. Berbagai pustaka seperti validasi \textit{form}, mengunggah file, berbagai macam teknologi autentifikasi bisa digunakan dan tersedia untuk Flask. Bahkan pustaka-pustaka pendukung tersebut lebih sering diperbarui dibandingkan dengan Flasknya sendiri.

\section{Nginx}
Nginx adalah sebuah perangkat lunak yang bisa digunakan untuk \textit{web server}, \textit{load balancer}, dan \textit{reverse proxy}. Nginx terkenal karena stabil, memiliki tingkat performa tinggi dan konsumsi sumber daya yang minim. Pada kasus saat terjadi koneksi dalam jumlah yang banyak secara bersamaan, penggunaan \textit{memory}, CPU, dan sumber daya sistem yang lain sangat kecil dan stabil. \cite{bab2-nginx}\\
\indent Nginx bisa digunakan untuk menyajikan kontent HTTP yang dinamis menggunakan FastCGI, SCGI untuk menangani scripts, aplikasi WSGI , dan bisa juga digunakan sebagai sebuah \textit{load balancer}. Nginx menggunakan \textit{asynchronous event-driven} untuk menangani permintaan. Dengan menggunakan model ini bisa, pengembang bisa melakukan predeksi kinerja Nginx saat terjadi jumlah permintaan yang banyak.

\section{Iptables} 
\textit{Firewall} merupakan sebuah mekanisme wajib \textit{access} kontrol antar jaringan ataupun antar sistem. \textit{Firewall} ini sangat penting karena bertujuan untuk memastikan keamanan dari sebuah jaringan. \textit{Firewall} dapat menjadi \textit{filter} yang sangat sederhana dan mudah digunakan, tetapi \textit{firewall} juga dapat menjadi \textit{filter} yang sangat penting bagi sebuah jalan keluar suatu jaringan. Prinsip dari penggunaan \textit{firewall} tetaplah sama, dimana penggunaannya untuk \textit{monitoring} dan \textit{filtering} semua pertukaran informasi di jaringan \textit{internal} dan juga di jaringan \textit{external}.

\textit{Netfilter} / Iptables merupakan sebuah sistem \textit{firewall} berbasis linux yang mempunyai fungsi yang sangat berguna. \textit{Netfilter} / \textit{iptables kernel} menggunakan sebuah mekanisme baru, bernama Iptables. \textit{Ipbtales} sendiri merupakan sebuah perangkat lunak atau alat yang dapat melakukan manajemen \textit{filter} dari sebuah paket yang ada pada suatu \textit{kernel}. Iptables mempunyai \textit{table} dan juga \textit{chain} dari masing-masing \textit{table}. \textit{Table} pada Iptables terdiri dari tiga, atau juga bisa disebut Iptables memiliki tiga fungsi utama, antara lain menjadi penyaring paket, mentranslasikan suatu alamat, dan melakuakn penghalusan paket seperti TTL, TOS, dan MARK. \cite{bab2-iptables}

\textit{Filter table} merupakan sebuah konfigurasi \textit{default} dari Iptables, dimana pada \textit{filter table} terdapat tiga \textit{chain}, antara lain \textit{chain} INPUT, FORWARD, dan OUTPUT. \textit{NAT table} berfungsi untuk merubah tujuan dari sumber dari sebuah paket. Pada \textit{NAT table} terdapat dua \textit{chain}, antara lain \textit{chain} PREROUTING dan POSTROUTING. \textit{Mangle table} berfungsi untuk menghaluskan paket atau juga dapat mengubah isi dari sebuah data kecuali IP \textit{address} dan \textit{port address}.Pada \textit{mangle table} terdapat dua \textit{chain}, antara lain POSTROUTING dan OUTPUT.

\section{MySQL}
MySQL adalah sebuah perangkat lunak terbuka untuk melakukan manajemen basis data SQL atau DBMS. MySQL ditulis dalam bahasa pemrograman C dan C++. MySQL merupakan salah satu perangkat lunak terbuka yang banyak disukai oleh pengembang dan digunakan dalam banyak aplikasi web. Parser SQL yang digunakan ditulis dalam bahasa pemrograman yacc. MySQL bekerja pada banyak \textit{platform}, seperti  FreeBSD, HP-UX, Linux, macOS, Microsoft Windows, NetBSD, OpenBSD, OpenSolaris, Oracle Solaris, dan SunOS. MySQL tersedia sebagai perangkat lunak gratis di bawah lesensi \textit{GNU General Public License} (GPL), tetapi juga tersedia lisensi komersial untuk kasus-kasus dimana penggunanya tidak cocok dengan penggunaan GPL. \cite{bab2-mysql}\\
\indent Setiap pengguna dapat secara bebas menggunakan MySQL, namun dengan batasan perangkat lunak tersebut tidak boleh dijadikan produk turunan yang bersifat komersial. MySQL sebenarnya merupakan turunan salah satu konsep utama dalam basis data yang telah ada sebelumnya, yaitu SQL (\textit{Structured Query Language}). SQL adalah sebuah konsep pengoperasian basis data, terutama untuk proses pemilihan atau seleksi dan pemasukan data, yang memungkinkan pengoperasian data dikerjakan dengan mudah.\\
\indent Kehandalan suatu sistem basis data dapat diketahui dari cara kerja pengoptimasiannya dalam melakukan proses perintah-perintah SQL yang dibuat oleh pengguna maupun program-program aplikasi yang memanfaatkannya. Sebagai \textit{server} basis data, MySQL mendukung operasi basis data transaksional maupun operasi basis data non-transaksional. Pada modus operasi non-transaksional, MySQL dapat dikatakan handal dalam hal unjuk kerja dibandingkan \textit{server} basis data kompetitor lainnya. Namun pada modus non-transaksional tidak ada jaminan atas reliabilitas terhadap data yang tersimpan, karenanya modus non-transaksional hanya cocok untuk jenis aplikasi yang tidak membutuhkan reliabilitas data seperti aplikasi blogging berbasis web (wordpress), CMS, dan sejenisnya. Untuk kebutuhan sistem yang ditujukan untuk bisnis sangat disarankan untuk menggunakan modus basis data transaksional, hanya saja sebagai konsekuensinya unjuk kerja MySQL pada modus transaksional tidak secepat unjuk kerja pada modus non-transaksional.

\section{Mitmproxy}

Mitmproxy adalah sebuah sebuah \textit{interception proxy} untuk HTTP dengan antarmuka pengguna \textit{console} yang ditulis dengan bahasa \textit{Python}. Mitmproxy merupakan sebuah perangkat lunak yang interaktif dimana Mitmproxy memungkinkan dapat memotong dan memodifikasi HTTP \textit{requests} atau \textit{response} dengan sangat cepat.

Mitmproxy adalah sebuah \textit{proxy} berkemampuan SSL yang berfungsi sebagai \textit{man-in-the-middle} untuk komunikasi HTTP dan HTTPS. Untuk dapat mengetahui atau memodifikasi komunikasi HTTPS, Mitmproxy berupra-pura menjadi \textit{server} ke \textit{client} dan \textit{client} ke server, sementara itu Mitmproxy diposisikan di tengah-tengah berfungsi untuk menerjemahkan lalu lintas dari keduanya. Mitmproxy menghasilkan sertifikat \textit{on-the-fly} untuk mengetahui \textit{client} agar percaya bahwa mereka berkomunikasi dengan \textit{server}. \cite{bab2-mitmproxy}

Pertama kali Mitmproxy dimulai, maka akan menghasilkan sertifikat SSL yang berada pada \texttt{~/.mitmproxy/cert.pem}. Sertifikat ini akan digunakan untuk \textit{browser-side}. Karena tidak akan cocok dengan \textit{domain} yang \textit{client} kunjungi, dan tidak akan memverifikasi terhadap otoritas sertifikasi, \textit{client} harus menambahkan pengecualian untuk setiap situs yang \textit{client} kunjungi. Permintaan SSL dicegat dengan hanya mengamsumsikan bahwa semua permintaan \texttt{CONNECT} adalah HTTPS. Sambungan dari \textit{browser} dibungkus SSL, dan kita membaca permintaan dengan berpura-pura menjadi \textit{server} yang menghubungkan.

\section{VirtualBox}
\textit{VirtuaBox} merupakan salah satu produk perangkat lunak yang sekarang dikembangkan oleh Oracle. Aplikasi ini pertama kali dikembangkan oleh perusahaan Jerman, Innotek GmbH. Februari 2008, Innotek GmbH diakusisi oleh Sun Micorsystems. Sun Microsystems kemudian juga diakuisisi oleh Oracle. VirtualBox berfungsi untuk melakukan virtualisasi sistem operasi. VirtualBox juga dapat digunakan untuk membuat virtualisasi jaringan komputer sederhana. Penggunaan VirtualBox ditargetkan untuk \textit{server}, desktop, dan penggunaan \textit{embedded}.

Berdasarkan jenis VMM yang ada, VirtualBox merpakan jenis \textit{hypervisor type 2}. VirtualBox sendiri memiliki berbagai macam kegunaan, diantaranya VirtualBox dapat memainkan semua sistem  operasi baik itu menggunakan windows, linux, atau turunan linux lainnya. VirtualBox juga dapat dipergunakan untuk mengujicoba OS baru. VirtualBox juga dapat digunakan sebgai media untuk membaut simulasi jaringan.

\section{Kontainer}
Kontainer merupakan alat untuk mempermudah mengemas dan mendistribusikan suatu hal dari satu tempat ke tempat lain. Sedangkan dalam konteks lingkungan linux, kontainer dapat diartikan sebagai alat yang dapat dipergunakan untuk memberikan sistem yang terisolasi pada \textit{level operating system} yang dijalankan pada satu indux linux \textit{kernel}. \cite{bab2-kontainer}

Untuk menangani perbedaan variabel \textit{system environment} dapat menjalankan \textit{command pipeline}, dan hal tersebut memungkinkan untuk terjadinya \textit{humman error} karena terlalu banyak dan rumitnya konfigurasi yang perlu disiapkan. Namun dengan bantuan kontainer, dapat membuat hal tersebut menjadi sangat mudah, dengan bantuan teknologi kontainer, satu induk dengan kernel tertentu dapat menjalnkan beberapa kontainer dengan masing-masing lingkungan yang terisolasi utuh, seperti halnya virtualisasi.

\subsection{Kontainer Berbasis Sistem Operasi}
Kontainer berbasis sistem operasi merupakan teknologi kontainer yang memperlakukan kontainer-kontainer didalamnya sebagai satu kesatuan sistem secara utuh seolah-olah dalam satu sistem operasi tersendiri secara tersionlasi.

\subsection{Kontainer Berbasis Aplikasi}
Kontainer berbasis aplikasi memanfaatkan \textit{hardware} dari induknya, kernelnya juga dari induknya, dan dia juga dapat memanfaatkan \textit{service-service} lain dari kontainer-kontainer lain yang berjalan pada sistem yang saling berhubungan. Hal ini memungkinkan kita untuk dapat \textit{mendeploy} banyak \textit{service} yang bergantung pada \textit{service-service} pendukungnya secara efisien karena tidak perlu menciptakan lingkungan sistem operasi sendiri serta tidak ada diplikasi \textit{service}.

\section{Docker}
Docker adalah sebuah aplikasi yang bersifat \textit{open source} yang berfungsi sebagai wadah untuk memasukkan sebuah perangkat lunak secara lengkap beserta semua hal yang dibutuhkan oleh perangkat lunak tersebut agar dapat berfungsi sebagaimana mestinya. Docker dapat dijalankan di berbagai sistem operasi, pengembang dapat dengan mudah menggunakan layanan Docker melalui \texttt{https://hub.docker.com} untuk mengunduh \textit{imaes} ataupun membuat \textit{images} yang diinginkan. \cite{bab2-docker}

\subsection{\textit{Docker Container}}
\textit{Docker container} atau kontainer Docker bisa dikatakan sebagai sebuah wadah atau tempat, dimana kontainer Docker ini dibuat dengan menggunakan \textit{docker image}. \cite{bab2-docker-container} Saat kontainer Docker dijalankan, maka akan terbentuk sebuah \textit{layer} di atas \textit{docker image}.Contohnya saat menggunakan \textit{image} Ubuntu, kemudian membuat sebuah kontainer Docker dari \textit{image} Ubuntu tersebut dengan nama mitmproxy-ubuntu. Setelah itu dilakukan pemasangan sebuah perangkat lunak, misalnya Mitmproxy, maka secara otomatis kontainer Docker mitmproxy-ubuntu akan berada di atas \textit{layer image} Ubuntu, dan diatasnya lagi merupakan \textit{layer mitmproxy} berada. \textit{Docker Kontainer} atau Kontainer Docker ke depannya dapat digunakan untuk menghasilkan sebuah \textit{docker images}. \textit{Docker images} yang dihasilkan dari kontainer Docker itu sendiri nantinya dapat digunakan kembali untuk membuat kontainer Docker yang lainnya.

\subsection{\textit{Docker Images}}
\textit{Docker images} adalah sebuah \textit{blueprint} atau rancangan dasar dari sebuah perangkat lunak berbasis Docker yang bersifat \textit{read-only}. \textit{Blueprint} ini sendiri merpakan sebuah sistem operasi atau sistem operasi yang telah dipasang berbagai perangkat lunak dan pustaka pendukung. \textit{Docker iamges} berfungsi untuk membuat kontainer Docker, dimana dengan menggunakan satu \textit{docker iamge} dapat dibuat lebih dari satu kontainer Docker. \textit{Docker image} sendiri dapat menyelesaikan permasalahan yang dikenal dengan "\textit{dependency hell}", dimana sulitnya untuk melengkapi dependensi sebuah perangkat lunak. Permasalahan tersebut dapat diselesaikan karena semua kebutuhan perangkat lunak sudah berada di dalamnya.

\subsection{\textit{Docker Registry}}
\textit{Docker Registry} adalah kumpulan dari berbagai macam \textit{docker image} yang bersifat tertutup maupun terbuka yang dapat diakses di \texttt{https://hub.docker.com/} atau dapat diakses pada \textit{server} sendiri. Dengan menggunakan \textit{docker registry}, seseorang dapat menggunakan \textit{docker image} yang telah dibuat oleh orang lainnya. Hal seperti ini dapat mempermudah seseorang untuk melakukan pengembangan dan jugatransfer aplikasi. \cite{bab2-docker-registry}
